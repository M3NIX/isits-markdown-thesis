%% Author: julian@ortel.tech
%% modified default pandoc template `pandoc -D latex`

% Options for packages loaded elsewhere
\PassOptionsToPackage{unicode}{hyperref}
\PassOptionsToPackage{hyphens}{url}

\documentclass[
  11pt,
  a4paper,
  openright,
  cleardoublepage=plain,
  parskip=half+, % comment this out if you do not want an empty half line between paragraphs, but please read the KomaScript Guide and search for parskip (around page 82): ftp://ftp.dante.de/pub/tex/macros/latex/contrib/koma-script/scrguide.pdf
]{scrreprt}

%% general stuff

% language
\usepackage{iflang}
\usepackage[ngerman]{babel}

% fonts
\usepackage{lmodern}
\usepackage{iftex}
\ifPDFTeX
  \usepackage[T1]{fontenc}
  \usepackage[utf8]{inputenc}
  \usepackage{textcomp} % provide euro and other symbols
\else % if luatex or xetex
  \usepackage{unicode-math}
  \defaultfontfeatures{Scale=MatchLowercase}
  \defaultfontfeatures[\rmfamily]{Ligatures=TeX,Scale=1}
\fi
% Use upquote if available, for straight quotes in verbatim environments
\IfFileExists{upquote.sty}{\usepackage{upquote}}{}
\IfFileExists{microtype.sty}{% use microtype if available
  \usepackage[]{microtype}
  \UseMicrotypeSet[protrusion]{basicmath} % disable protrusion for tt fonts
}{}
\makeatletter
\@ifundefined{KOMAClassName}{% if non-KOMA class
  \IfFileExists{parskip.sty}{%
    \usepackage{parskip}
  }{% else
    \setlength{\parindent}{0pt}
    \setlength{\parskip}{6pt plus 2pt minus 1pt}}
}{% if KOMA class
  \KOMAoptions{parskip=half}}
\makeatother
\usepackage{xcolor}

% code blocks
\definecolor{colKeys}{rgb}{0,0,1}
\definecolor{colIdentifier}{rgb}{0,0,0}
\definecolor{colComments}{rgb}{0.5, 0.5, 0.5}
\definecolor{colString}{rgb}{0,0.5,0}
\usepackage{listings}
\IfLanguageName{english}{
  \renewcommand{\lstlistingname}{Algorithm}
}{}
\IfLanguageName{ngerman}{
  \renewcommand{\lstlistingname}{Algorithmus}
}{}
\lstset{%
    float=hbp,%
    basicstyle=\ttfamily\small, %
    identifierstyle=\color{colIdentifier}, %
    keywordstyle=\color{colKeys}, %
    stringstyle=\color{colString}, %
    commentstyle=\color{colComments}, %
    columns=flexible, %
    tabsize=2, %
    frame=single, %
    extendedchars=true, %
    showspaces=false, %
    showstringspaces=false, %
    numberstyle=\tiny, %
    breaklines=true, %
    backgroundcolor=, %
    breakautoindent=true, %
    captionpos=b, %
    numbers=left %
}
\newcommand{\passthrough}[1]{#1}
\lstset{defaultdialect=[5.3]Lua}
\lstset{defaultdialect=[x86masm]Assembler}

% tables
\usepackage{longtable,booktabs,array}
\usepackage{calc} % for calculating minipage widths
% Correct order of tables after \paragraph or \subparagraph
\usepackage{etoolbox}
\makeatletter
\patchcmd\longtable{\par}{\if@noskipsec\mbox{}\fi\par}{}{}
\makeatother
% Allow footnotes in longtable head/foot
\IfFileExists{footnotehyper.sty}{\usepackage{footnotehyper}}{\usepackage{footnote}}
\makesavenoteenv{longtable}

% images
\usepackage{graphicx}
\usepackage{float}
\makeatletter
\def\maxwidth{\ifdim\Gin@nat@width>\linewidth\linewidth\else\Gin@nat@width\fi}
\def\maxheight{\ifdim\Gin@nat@height>\textheight\textheight\else\Gin@nat@height\fi}
\makeatother
% Scale images if necessary, so that they will not overflow the page
% margins by default, and it is still possible to overwrite the defaults
% using explicit options in \includegraphics[width, height, ...]{}
% \setkeys{Gin}{width=\maxwidth,height=\maxheight,keepaspectratio}
\setkeys{Gin}{keepaspectratio}
\makeatletter
\def\fps@figure{H} % Set default figure placement to H
\makeatother

% misc

% headings / pagestyle
\usepackage[bottom]{footmisc} % position footnotes at the bottom of the document
\setlength{\emergencystretch}{3em} % prevent overfull lines
\providecommand{\tightlist}{%
  \setlength{\itemsep}{0pt}\setlength{\parskip}{0pt}}
\setcounter{secnumdepth}{5}

% bibliography
\usepackage[square,numbers]{natbib}
\bibliographystyle{dinat}
\IfFileExists{bookmark.sty}{\usepackage{bookmark}}{\usepackage{hyperref}}
\IfFileExists{xurl.sty}{\usepackage{xurl}}{} % add URL line breaks if available
\urlstyle{same} % disable monospaced font for URLs
\usepackage{tikz}

% pdf settings
\hypersetup{
  pdftitle={Mein toller Titel für die Masterarbeit},
  pdfauthor={Julian Ortel},
  pdflang={ngerman},
  hidelinks,
  pdfcreator={LaTeX via pandoc}
}

\usepackage{acro}
\DeclareAcronym{isits}{
short = isits,
long = International School of IT Security
}

% BEGIN: variable defininiton (e.g. for the titlepage)
\title{Mein toller Titel für die Masterarbeit}


\author{Julian Ortel}

\date{01.02.2034}

\newcommand{\thtype}{Masterarbeit}
% END: define some variables

% BEGIN: latex document
\begin{document}

\begin{titlepage}
\pagenumbering{gobble}
\makeatletter

\enlargethispage{3cm}

\begin{tikzpicture}[remember picture,overlay]
% \node[shift={(14cm,-23cm)},opacity=1] {\includegraphics[scale=0.625]{data/logo/hgi-logo}};
\node[shift={(14cm,-23cm)},opacity=1] {\includegraphics[scale=0.15]{data/logo/isits-logo}};
\node[shift={(13cm,1.22cm)},opacity=1] {\includegraphics[scale=1.8]{data/logo/rub-logo}};
\end{tikzpicture}
 
\vspace*{10cm}
\begin{minipage}[b]{1\linewidth}
	\sffamily
  	\hspace{-17.2mm}\includegraphics[scale=1.0]{data/logo/rub-slogan}\\

   	\textbf{\LARGE {\@title}}\\
  
  	\Large{\@author}\\
		
  	\vspace{3cm}
  	\normalsize{
   	  \thtype\@~~--~~\@date\@\\
			\IfLanguageName{english}{
			Chair for System Security.\\
			}{}
			\IfLanguageName{ngerman}{
		  Lehrstuhl für Systemsicherheit\\
			}{}
		}
		\newline
	  \normalsize{
	    \begin{tabular}{@{}ll@{}}
			  \IfLanguageName{english}{
				1st Supervisor: & Prof.~Dr.~Thorsten~Holz\\
				2nd Supervisor: & Prof.~Dr.~Someone~Else\\
				Advisor: & Another Guy, Maybe Another\\
			  }{}
			  \IfLanguageName{ngerman}{
				1. Prüfer: & Prof.~Dr.~Thorsten~Holz\\
				2. Prüfer: & Prof.~Dr.~Someone~Else\\
				Betreuung: & Another Guy, Maybe Another\\
			  }{}
	      \end{tabular}
	  }
\end{minipage}

\makeatother
\end{titlepage}

\chapter*{Eigenständigkeitserklärung}

Hiermit bestätige ich, dass ich die vorliegende Arbeit selbständig verfasst und keine
anderen als die angegebenen Hilfsmittel benutzt habe. Die Stellen der Arbeit, die dem Wortlaut oder dem Sinn nach anderen Werken (dazu zählen auch Internetquellen) entnommen sind, wurden unter Angabe der Quelle kenntlich gemacht.
\newline

\hrulefill \hfill \hrulefill

Ort, Datum\hfill Unterschrift

\pagenumbering{gobble}
\newpage



 % table of contents
\pagenumbering{gobble} % disable numbering style
\tableofcontents
\newpage


\pagenumbering{arabic} % start Arabic numbering style
\chapter*{Abstract}\label{abstract}
\addcontentsline{toc}{chapter}{Abstract}

Lorem ipsum dolor sit amet, consectetur adipiscing elit, sed do eiusmod
tempor incididunt ut labore et dolore magna aliqua. Convallis aenean et
tortor at risus. Risus ultricies tristique nulla aliquet enim tortor at
auctor. Molestie ac feugiat sed lectus vestibulum mattis ullamcorper.
Lectus arcu bibendum at varius vel pharetra vel turpis nunc. Cursus
mattis molestie a iaculis at erat. Amet mattis vulputate enim nulla
aliquet porttitor. Mauris augue neque gravida in fermentum et
sollicitudin ac. Vel turpis nunc eget lorem dolor. Arcu vitae elementum
curabitur vitae nunc sed. Diam quis enim lobortis scelerisque fermentum
dui faucibus in ornare. Sagittis aliquam malesuada bibendum arcu vitae
elementum curabitur vitae. Sem integer vitae justo eget magna.

Gravida cum sociis natoque penatibus et. Elementum curabitur vitae nunc
sed. Nunc consequat interdum varius sit amet mattis. Sed euismod nisi
porta lorem. Volutpat lacus laoreet non curabitur gravida. Pulvinar
etiam non quam lacus suspendisse. Vulputate odio ut enim blandit. Sed
felis eget velit aliquet sagittis id consectetur purus ut. Et leo duis
ut diam quam nulla porttitor. Ut eu sem integer vitae justo eget magna
fermentum iaculis. Tellus in metus vulputate eu scelerisque. Elit duis
tristique sollicitudin nibh sit amet. Convallis a cras semper auctor
neque vitae. Mauris pharetra et ultrices neque. Sagittis eu volutpat
odio facilisis mauris. Facilisi nullam vehicula ipsum a arcu cursus
vitae. Accumsan sit amet nulla facilisi morbi tempus. Vitae sapien
pellentesque habitant morbi tristique senectus et.

Enim facilisis gravida neque convallis. Mauris ultrices eros in cursus
turpis. Non quam lacus suspendisse faucibus interdum posuere. Aliquet
porttitor lacus luctus accumsan tortor posuere ac. Eu sem integer vitae
justo eget magna. Eu nisl nunc mi ipsum faucibus. Pulvinar mattis nunc
sed blandit libero volutpat. Dictum non consectetur a erat nam. Fusce ut
placerat orci nulla pellentesque dignissim. Tincidunt dui ut ornare
lectus sit amet est. Neque laoreet suspendisse interdum consectetur
libero id faucibus nisl tincidunt. Porta nibh venenatis cras sed.
Praesent semper feugiat nibh sed pulvinar proin gravida hendrerit
lectus. Egestas tellus rutrum tellus pellentesque eu tincidunt. Non
pulvinar neque laoreet suspendisse interdum consectetur libero id
faucibus. Feugiat vivamus at augue eget. Hendrerit gravida rutrum
quisque non tellus orci. Pellentesque elit ullamcorper dignissim cras
tincidunt lobortis feugiat. Rhoncus urna neque viverra justo nec
ultrices.

\chapter{Überschrift}\label{uxfcberschrift}

Te \emph{concepit} pollice fugit vias alumno \textbf{oras} quam potest
\href{http://example.com\#rursus}{rursus} optat. Non evadere orbem
equorum, spatiis, vel pede inter si.

\begin{enumerate}
\def\labelenumi{\arabic{enumi}.}
\tightlist
\item
  De neque iura aquis
\item
  Frangitur gaudia mihi eo umor terrae quos
\item
  Recens diffudit ille tantum
\end{enumerate}

Tamen condeturque saxa Pallorque num et ferarum promittis inveni lilia
iuvencae adessent arbor. Florente perque at condeturque saxa et ferarum
promittis tendebat. Armos nisi obortas refugit me.

\begin{quote}
Et nepotes poterat, se qui. Euntem ego pater desuetaque aethera
Maeandri, et \href{http://example.com\#Dardanio_geminaque}{Dardanio
geminaque} cernit. Lassaque poenas nec, manifesta \(\pi r^2\) mirantia
captivarum prohibebant scelerato gradus unusque dura.
\end{quote}

\begin{itemize}
\tightlist
\item
  Permulcens flebile simul
\item
  Iura tum nepotis causa motus diva virtus Acrota. Tamen condeturque
  saxa Pallorque num et ferarum promittis inveni lilia iuvencae adessent
  arbor. Florente perque at ire arcum.
\end{itemize}

\section{Unterüberschrift}\label{unteruxfcberschrift}

Lorem ipsum dolor sit amet, consectetur adipiscing elit, sed do eiusmod
tempor incididunt ut labore et dolore magna aliqua. Viverra aliquet eget
sit amet tellus. Arcu dictum varius duis at.

\subsection{Unterunterüberschrift}\label{unterunteruxfcberschrift}

Lorem ipsum dolor sit amet, consectetur adipiscing elit, sed do eiusmod
tempor incididunt ut labore et dolore magna aliqua. Viverra aliquet eget
sit amet tellus. Arcu dictum varius duis at.

\chapter{Beispielinhalte}\label{beispielinhalte}

\section{Tabellen}\label{tabellen}

\begin{longtable}[]{@{}lll@{}}
\toprule\noalign{}
Header A & Header B & Header C \\
\midrule\noalign{}
\endhead
\bottomrule\noalign{}
\endlastfoot
Value A1 & Value B1 & Value C1 \\
Value A2 & Value B2 & Value C2 \\
Value A3 & Value B3 & Value C3 \\
Value A4 & Value B4 & Value C4 \\
\end{longtable}

\begin{longtable}[]{@{}lll@{}}
\caption{Tabelle mit Überschrift \label{tbl:short-table}}\tabularnewline
\toprule\noalign{}
Header A & Header B & Header C \\
\midrule\noalign{}
\endfirsthead
\toprule\noalign{}
Header A & Header B & Header C \\
\midrule\noalign{}
\endhead
\bottomrule\noalign{}
\endlastfoot
Value A1 & Value B1 & Value C1 \\
Value A2 & Value B2 & Value C2 \\
Value A3 & Value B3 & Value C3 \\
Value A4 & Value B4 & Value C4 \\
\end{longtable}

\begin{longtable}[]{@{}
  >{\raggedright\arraybackslash}p{(\columnwidth - 2\tabcolsep) * \real{0.1266}}
  >{\raggedright\arraybackslash}p{(\columnwidth - 2\tabcolsep) * \real{0.8734}}@{}}
\caption{Tabelle mit langen Zeilen
\label{tbl:long-table}}\tabularnewline
\toprule\noalign{}
\begin{minipage}[b]{\linewidth}\raggedright
Header A
\end{minipage} & \begin{minipage}[b]{\linewidth}\raggedright
Header B
\end{minipage} \\
\midrule\noalign{}
\endfirsthead
\toprule\noalign{}
\begin{minipage}[b]{\linewidth}\raggedright
Header A
\end{minipage} & \begin{minipage}[b]{\linewidth}\raggedright
Header B
\end{minipage} \\
\midrule\noalign{}
\endhead
\bottomrule\noalign{}
\endlastfoot
Value A1 & ut etiam sit amet nisl purus in mollis nunc sed id semper
risus in hendrerit gravida rutrum quisque non tellus orci ac auctor
augue mauris augue neque gravida in fermentum \\
Value A2 & ut etiam sit amet nisl purus in mollis nunc sed id semper
risus in hendrerit gravida rutrum quisque non tellus orci ac auctor
augue mauris augue neque gravida in fermentum \\
Value A3 & ut etiam sit amet nisl purus in mollis nunc sed id semper
risus in hendrerit gravida rutrum quisque non tellus orci ac auctor
augue mauris augue neque gravida in fermentum \\
Value A4 & ut etiam sit amet nisl purus in mollis nunc sed id semper
risus in hendrerit gravida rutrum quisque non tellus orci ac auctor
augue mauris augue neque gravida in fermentum \\
\end{longtable}

\section{Bilder}\label{bilder}

\begin{figure}
\centering
\includegraphics[width=0.5\textwidth,height=\textheight]{././data/logo/isits-logo.jpg}
\caption{Logo der International School of IT
Security}\label{fig:logo-isits}
\end{figure}

\begin{figure}
\centering
\includegraphics[width=0.1\textwidth,height=\textheight]{././data/logo/isits-logo.jpg}
\caption{Logo der International School of IT Security -
klein}\label{fig:logo-isits-small}
\end{figure}

\section{Code}\label{code}

\begin{lstlisting}[language=Python, caption={Fibonacci-Algorithmus in Python}, label={alg:fibonacci-python}]
# Function for nth Fibonacci number
def Fibonacci(n):
   
    # Check if input is 0 then it will
    # print incorrect input
    if n < 0:
        print("Incorrect input")
 
    # Check if n is 0
    # then it will return 0
    elif n == 0:
        return 0
 
    # Check if n is 1,2
    # it will return 1
    elif n == 1 or n == 2:
        return 1
 
    else:
        return Fibonacci(n-1) + Fibonacci(n-2)
 
# Driver Program
print(Fibonacci(9))
\end{lstlisting}

Es ist sogar möglich Code aus anderen Dateien zu inkludieren:

\begin{lstlisting}[caption=Makefile, label={alg:makefile}]
docker-build:
	docker build -t $$(id -un)/pandoc-latex .

docker-pdf:
	docker run --rm --volume "$$(pwd):/data" --user $$(id -u):$$(id -g) $$(id -un)/pandoc-latex make pdf

docker-tex:
	docker run --rm --volume "$$(pwd):/data" --user $$(id -u):$$(id -g) $$(id -un)/pandoc-latex make tex

pdf: tex
	latexmk thesis.tex -pdf

tex: clean
	pandoc -d pandoc/config.yml markdown/*.md -o thesis.tex

.PHONY: clean
clean:
	find . -name 'thesis.*' -a ! -name '*.pdf' -a ! -name '*.tex' -exec rm {} \;
\end{lstlisting}

\section{Querverweise}\label{querverweise}

Querverweise können mit \passthrough{\lstinline!\\ref\{label\}!} oder
\passthrough{\lstinline!\\autoref\{label\}!} erstellt werden.

Beispiele:

\begin{itemize}
\tightlist
\item
  Querverweis zu \autoref{fazit} oder nur der Nummer (\ref{fazit}) vom
  Kapitel
\item
  Querverweis zu \autoref{bilder} oder nur der Nummer (\ref{bilder}) vom
  Abschnitt
\item
  Querverweis zu \autoref{fig:logo-isits} oder nur der Nummer
  (\ref{fig:logo-isits}) vom Bild
\item
  Querverweis zu \autoref{alg:fibonacci-python} oder nur der Nummer
  (\ref{alg:fibonacci-python}) vom Codeblock
\item
  Querverweis zu \autoref{tbl:long-table} oder nur der Nummer
  (\ref{tbl:long-table}) von der Tabelle
\item
  Querverweis zu \autoref{eq:neighbor-propability} oder nur der Nummer
  (\ref{eq:neighbor-propability}) von der Formel
\end{itemize}

\section{Quellen}\label{quellen}

Quellen werden in der Datei
\passthrough{\lstinline!data/references.bib!} gepflegt und konnen im
Text mit \passthrough{\lstinline![@label]!} referenziert werden. Zum
Beispiel stammt der Code für \autoref{alg:fibonacci-python} von
\passthrough{\lstinline!geeksforgeek.org!} \citep{fibonacci-python}.

\section{Formeln}\label{formeln}

Mit \passthrough{\lstinline!$!} umschlossene Strings werden in der Zeile
als Formel interpretiert: \(y = mx +b\)

Eine komplette Formel kann als Block mit \passthrough{\lstinline!$$!}
umschlossen werden: \[
x = {-b \pm \sqrt{b^2-4ac} \over 2a}
\]

Die Verwendung von Latex innerhalb des Markdowns ist auch möglich:
\begin{equation}\label{eq:neighbor-propability}
    p_{ij}(t) = \frac{\ell_j(t) - \ell_i(t)}{\sum_{k \in N_i(t)}^{} \ell_k(t) - \ell_i(t)}
\end{equation}

\section{Sonstiges}\label{sonstiges}

Fußnoten\footnote{https://github.com/M3NIX/isits-markdown-thesis} sind
ebenfalls im Markdown-Style möglich.

Abkürzungen wie \ac{isits} können auch verwendet werden und werden bei
der ersten Verwendung automatisch ausgeschrieben und bei der zweiten
Verwendung (hier: \ac{isits}) nicht mehr ausgeschrieben. Definiert
werden diese in \passthrough{\lstinline!data/acronyms.yml!}.

Pfeile werden durch den Pandoc-Filter
\passthrough{\lstinline!pandoc/pandoc-latex-arrows.py!} automatisch
umgewandelt:

\begin{longtable}[]{@{}ccc@{}}
\toprule\noalign{}
markdown & latex & pdf \\
\midrule\noalign{}
\endhead
\bottomrule\noalign{}
\endlastfoot
\passthrough{\lstinline!->!} & \passthrough{\lstinline!$\\rightarrow$!}
& $\rightarrow$ \\
\passthrough{\lstinline!<-!} & \passthrough{\lstinline!$\\leftarrow$!} &
$\leftarrow$ \\
\passthrough{\lstinline!<->!} &
\passthrough{\lstinline!$\\leftrightarrow$!} & $\leftrightarrow$ \\
\passthrough{\lstinline!=>!} & \passthrough{\lstinline!$\\Rightarrow$!}
& $\Rightarrow$ \\
\passthrough{\lstinline!<=!} & \passthrough{\lstinline!$\\Leftarrow$!} &
$\Leftarrow$ \\
\passthrough{\lstinline!<=>!} &
\passthrough{\lstinline!$\\Leftrightarrow$!} & $\Leftrightarrow$ \\
\passthrough{\lstinline!==>!} &
\passthrough{\lstinline!$\\Longrightarrow$!} & $\Longrightarrow$ \\
\passthrough{\lstinline!<==!} &
\passthrough{\lstinline!$\\Longleftarrow$!} & $\Longleftarrow$ \\
\passthrough{\lstinline!<==>!} &
\passthrough{\lstinline!$\\Longleftrightarrow$!} &
$\Longleftrightarrow$ \\
\end{longtable}

\chapter{Fazit}\label{fazit}

Lorem ipsum dolor sit amet, consectetur adipiscing elit, sed do eiusmod
tempor incididunt ut labore et dolore magna aliqua. Convallis aenean et
tortor at risus. Risus ultricies tristique nulla aliquet enim tortor at
auctor. Molestie ac feugiat sed lectus vestibulum mattis ullamcorper.
Lectus arcu bibendum at varius vel pharetra vel turpis nunc. Cursus
mattis molestie a iaculis at erat. Amet mattis vulputate enim nulla
aliquet porttitor. Mauris augue neque gravida in fermentum et
sollicitudin ac. Vel turpis nunc eget lorem dolor. Arcu vitae elementum
curabitur vitae nunc sed. Diam quis enim lobortis scelerisque fermentum
dui faucibus in ornare. Sagittis aliquam malesuada bibendum arcu vitae
elementum curabitur vitae. Sem integer vitae justo eget magna.

Gravida cum sociis natoque penatibus et. Elementum curabitur vitae nunc
sed. Nunc consequat interdum varius sit amet mattis. Sed euismod nisi
porta lorem. Volutpat lacus laoreet non curabitur gravida. Pulvinar
etiam non quam lacus suspendisse. Vulputate odio ut enim blandit. Sed
felis eget velit aliquet sagittis id consectetur purus ut. Et leo duis
ut diam quam nulla porttitor. Ut eu sem integer vitae justo eget magna
fermentum iaculis. Tellus in metus vulputate eu scelerisque. Elit duis
tristique sollicitudin nibh sit amet. Convallis a cras semper auctor
neque vitae. Mauris pharetra et ultrices neque. Sagittis eu volutpat
odio facilisis mauris. Facilisi nullam vehicula ipsum a arcu cursus
vitae. Accumsan sit amet nulla facilisi morbi tempus. Vitae sapien
pellentesque habitant morbi tristique senectus et.

Enim facilisis gravida neque convallis. Mauris ultrices eros in cursus
turpis. Non quam lacus suspendisse faucibus interdum posuere. Aliquet
porttitor lacus luctus accumsan tortor posuere ac. Eu sem integer vitae
justo eget magna. Eu nisl nunc mi ipsum faucibus. Pulvinar mattis nunc
sed blandit libero volutpat. Dictum non consectetur a erat nam. Fusce ut
placerat orci nulla pellentesque dignissim. Tincidunt dui ut ornare
lectus sit amet est. Neque laoreet suspendisse interdum consectetur
libero id faucibus nisl tincidunt. Porta nibh venenatis cras sed.
Praesent semper feugiat nibh sed pulvinar proin gravida hendrerit
lectus. Egestas tellus rutrum tellus pellentesque eu tincidunt. Non
pulvinar neque laoreet suspendisse interdum consectetur libero id
faucibus. Feugiat vivamus at augue eget. Hendrerit gravida rutrum
quisque non tellus orci. Pellentesque elit ullamcorper dignissim cras
tincidunt lobortis feugiat. Rhoncus urna neque viverra justo nec
ultrices.
\newpage

\pagenumbering{roman} % start Roman numbering style

 % list of figures
\phantomsection
\listoffigures
\addcontentsline{toc}{chapter}{\listfigurename}
\newpage


 % list of tables
\phantomsection
\listoftables
\addcontentsline{toc}{chapter}{\listtablename}
\newpage


 % list of tables
\phantomsection
\IfLanguageName{english}{
  \renewcommand*{\lstlistlistingname}{List of Algorithms}
}{}
\IfLanguageName{ngerman}{
  \renewcommand*{\lstlistlistingname}{Algorithmenverzeichnis}
}{}
\lstlistoflistings
\addcontentsline{toc}{chapter}{\lstlistlistingname}
\newpage


\phantomsection
\IfLanguageName{ngerman}{
  \renewcommand*{\acrolistname}{Abkürzungsverzeichnis}
}{}
\printacronyms
\addcontentsline{toc}{chapter}{\acrolistname}
\newpage

 % list of bibliography
\phantomsection
 % list all references, even when not used
\nocite{*}

\bibliography{./data/references.bib}
\addcontentsline{toc}{chapter}{\bibname}
\newpage



\end{document}
% END: latex document
